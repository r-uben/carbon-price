16 indicators (Table 7 in \cite{baumeister2020energy}). Variable selection is guided by four principles:

\begin{enumerate}[wide, itemsep=0cm, topsep=0cm]
	\item The different categories of data in order to span multiple dimensions of the global economy. Eight categories: real economic activity, commodity prices, financial indicators, transportation, uncertainty, expectations, weather, and energy-related measures.
	\item Each individual variable should matter for energy demand on economic grounds.
	\item It should have the broades possible coverage gepgraphically, conceptually and in time.
	
	\noindent \textbf{Comment:} It may not be necessary in our setting, because we have to focus in Europe and it may be more interesting to capture the particularities of Europe.
	\item The number of variables should be kept at a manageable size to ensure that the dataset can be easily updated in a real-time setting.
	
	For example: EIA's December 2019 \emph{Short-Term Energy outlook} considers developments of US real GDP, China's Purchasing Managers' Index, the S\& P 500 equity index, and the \textbf{copper-to-gold ratio as a measure of market sentiment} on global economic growth as important factors in their assessment of energy markets.
\end{enumerate}