It was established in 2015 by the EU as part of EU ETS (substantially revised in 2018) and it's used to limit TNAC (total number of allowances in circulation).
\begin{enumerate}[wide, itemsep=0cm, topsep=0cm, label=$\bullet$]
	\item If banked EUAs (equivalently, TNAC) exceed 833 Mt CO$_2$, the number of auctioned EUAs next year is reduced.  The EUAs that are not auctioned are instead moved into the MSR. 24\% of (833 + $\varepsilon$) Mt CO$_2$ (banked EUAs) enter the MSR (12\% as of 2024). 
	\item If banked EUAs drop below 400 Mt CO$_2$, then EUAs corresponding to 100 Mt CO$_2$ are taken out of the MSR (and added to the auctioned volumes next year).
\end{enumerate} 
This is the very basic frame. As we have already mentioned, the MSR was revised in 2018. In that year, EU ETS let cancel allowances in the MSR when the number of EUAs in the MSR exceeds the number of auctioned EUAs of the previous year. All EUAs exceding this treshold are \emph{permanently} cancelled. This imply a reduction on the cumulative cap on emissions in the EU ETS. Hence, cumulative supply of EUAs has become \emph{endogenous} \cite{perino2018new,gerlagh2020covid}.

MSR cancels the perturbations of EUA demand \cite{gerlagh2020covid}, i.e., it stabilises the market when theres a drop in demand. However, the extent to which the market is stabilised is very sensitive to long term expectations, even more so than before the MSR. 

The EU ETS was constructed as a cap-and-trade system\footnote{See \url{https://www.edf.org/climate/how-cap-and-trade-works}} because certainty over quantities was perceived as more important than certainty over prices. MSR does stabilise prices, but it also opens the door to manipulation by unregulated parties.

\subsection{Example from \cite{perino2018new}}

If on 31st December of a given year there are more than 833 million allowances in circulation (banked for future use), then each subsequent year the number of allowances auction is reduced by 24\% (12\% after 2024) of the size of the bank. This continues annually until the bank drops below the threshold. Allowances withheld from auctioning are placed in the MSR\footnote{The MSR has been taken allowances for the first time in 2019 based on the total number of allowances that were in circulation at the end of 2017.}. Once the number of allowances banked drops below 400 million, in each subsequent year the MSR releases 100 milliion allowances via auctions until the MSR is empty. The MSR is seeded with 900 million backloaded allowances and some 550-700 million unallocated allowances from Phase 3. 

From 2023, the upper bound is moving. The MSR may hold \emph{only} as many allowances as were auctioned in the previous year, about 57\% of the annual cap. All allowances over and above this moving threshold are cancelled. Any allowance added to the MSR over and above the initial seeding gets cancelled in 2024, or thereafter. According to \cite{perino2017eu}, about 1.7 billioin allowances will be cancelled as a result.

The idea of this cancellation system is to \emph{puncture the waterbed}.

The number of allowances removed depends on two variables:
\begin{enumerate}[wide, itemsep=0cm, topsep=0cm, labelindent=2cm]
	\item The total number of allowances placed into the MSR (which in turn depends on the year the number of banked allowances drops below 833 million). This variables determine the maximum umber of allowances at risk of being cancelled.
	\item The point in time at which the number of allowances banked by firms drops below 400 million. This variable determines how many of those are spared that fate.
\end{enumerate}

Note that the puncture is retroactive, because the number of allowances placed into the MSR depends on the number of banked allowances at the end of 2017 and thereafter. Banking between phases was introduced in Phase 2 (2008-2012) and ever since the number of banked allowances has remained strictly positive.

It's important to take into account the impact on the optimal policy mix (policies overlapping the EU ETs) consisting in, say, supporting renewable energies, UK carbon tax, pphase out of coal, energy efficiency measures and voluntary abatement efforts. All of them change the number of allowances issued in the long run. For example, if a coal-fired power plan is forced to shut-down, then it induces a reduction in the demand for allowances and hence additional allowances  are banked for future use, which increases the number ending up in the MSR and eventually being cancelled (according to the new rulses).

