\begin{enumerate}[wide, itemsep=0cm, topsep=0cm, label=\textbf{\arabic{enumi}.}]
	\item ``Carbon pricing has significant effects on emissions and the economy''.
	
	\item There's a project of extending the carbon market to buildings and transportation. For the moment it's only for a couple of sectors \cite{eu2021carbon}.
	
	\item Heterogeneities (on the effects of carbon pricing on households). First, direct effect through enery prices (which depends ont he energy expenditure share, which is highly heterogeneous across households). Second, 
	
	\item There are some identified factors reveal that in the European and American market, as the energy consumptions of coal, oil, and natural gas directly relate to carbon emissions, the situations could affect the allocation of carbon emission alllowance. Thus, the unpredictable energy price fluctuations can affect the market demands on carbon emission allowance \cite{wen2022china, chevalier2012correlations}
	
	\item When the (macro) economy is prosperous, firms would expand production, which implies an increase in carbon emissions and demanding for more carbon allowances (because more carbon has been emitted), then to drive up the carbon prices. When the macro economy is in a downturn, firms would reduce production activities and carbon emission demands are correspondingly less, then to make lower carbon price. Sudden and unexpected events would also generate market and price fluctuations. \cite{wen2022china}
	
	\item Paper by \cite{wen2022china} adopts the economocy and policy uncertainty index in China to measure the uncertainty impacts on carbon markets. 
	
	\item Carbon price fluctuation is related with environment factor, such as the air quality and weather \cite{perera2020quantifying}.
	
	\item Dynamic connectedness measurement methodology in \cite{wen2022china} instead of static and multi regression models to study driving factors of China carbon price fluctiations. The main complain is that those people only rely on statistic significance and they do not wuantify the effects and consider the dynamics in sufficient details. 
	
	\item The carbon prices are mainly affected by the previous market prices \cite{wen2022china} in three Chinese carbon markets.
	
	\item ``Our empirical results suggest different carbon market should pay more attention to the policy utilisation in constructing process.'' \cite[p.8]{wen2022china}
	
	\item In the early phase of the EU ETS, there was an oversupply of carbon allowances, and therefore, carbon price was maintained at a low level. \cite{dong2022exploring}
	
	\item It may be to check the supply-side reform measures the European Comission has taken or will take \cite{hintermayer2020reforms}.
	
	\item To ensure the carbon price could be used as a signal to solve the \emph{oversupply of allowances}, the European Comission has promulgated a series of supply-side reform measures.
	\begin{enumerate}[wide, itemsep=0cm, topsep=0cm, label=\textbf{(\alph{enumii})}, labelindent=1cm]
		\item The government has implemented a linear decrease in the total quota (total number of allowances ahs been determined the beginning of the third stage and an allocation plan for the carbon allowances \emph{in each member state} has done). $-1.74\%$ per year. 
		\item Discounted action policy was implemented. In particular, the \emph{market stabilisation reserve mechanism}. When the number of quotas circulating in the market is too large, the European Comission incorporates part of the future auction quotas into the market reserve. Conversely, if the number of circulating quotas is too small, the market reserve mechanism will release carbon allowances. \cite{richstein2015market} 
	\end{enumerate}
	\item \textbf{Important:} Especially because of the operation of the market stability reserve mechanism, the supplu flexibility of carbon allowances has been significantly improved and the problem of oversupply has been alleviated. \textbf{However}, this had cause the sharp increase of the carbon price (at around \euro$25/\text{ton}$) \cite{dong2022exploring}.
	\item COVID-19 PANDEMIC: prolonged lock-downs have provoked the social demand to drop sharply. That has generated instability \emph{in the carbon market}. For instance, during the initial large outbreak in Italy, EU carbon price fell sharply in short time: from \euro25 to \euro15/ton. Under the influence of the market reserve mechanism, the carbon price rebounded briefly. \cite{gerlagh2020covid}
	\item If carbon price remains low for a long time, the confidence of market participants will be weaken, i.e., affecting the efficiency of emission reductions.
	\item Carbon price is the core driving factor in the operation of an emissions trading system \cite{dong2022exploring}.
	\item Carbon price is the core driving factor in the operation of an emissions trading systems. If its fluctuations are too severe, they will directly affect the realisation of the EU's emission reduction targets, which will not be conducive to the sustainable development of society.
	\item The price of carbon emission rights embodies the relative relationship between carbon allowance supply and demand.
	\item The supply of allowances is determined by government policies. When carbon emission market is establisedh, if the government sets a total number of carbon allowances that is too high, companies will obtain too many carbon allowances. If a free allocation policy is established, then the supply will be greater than the demand. Little incentive for companies to cut their own emissions. Hence, carbon price remains depressed. Also government announcements on the carbon market may have a significant impact on carbon price (adjustments), i.e., carbon market is very sensitive to related event.
	\item Energy price, macroeconomic development, major social events, and other factors will affect the demand for carbon allowances, causing ulterior fluctuations in carbon price. Energy price is the most important factor affecting carbon price trends.
	\item Both carbon price and the degree of emission reduction will be affected by economic crisis.
	\item The carbon finance market currently contains a large number of innovative carbon allowance financial derivatives, such as carbon futures, carbon options and carbon forwards (SPECULATION).
	\item Most studies have focused on only one or several factors, such as the economic situation, energy price, and quota reforms, \textbf{with little attention given to the impact of major social events on carbon price}.
	\item Metoddology of \cite{dong2022exploring}: Bai-Peron (B-P) structural break test to examine the structural changes in carbon price during the pandemic, and through the introduction of dummy variables, a comprehensive analysis of the mechanism of carbon price drivers during the COVID-19 pandemic was conducted. Other studies have used OLS methid to perform multiple linear regression (autocorrelation and heteroscedasticity problems). The Newey-West estimation method adopted overcomes  the problems of heteroscedasticity and autocorrelation,
	\item Negative correlation between carbon price and the macroeconomic situation in the short term \cite{dong2022exploring}.
	\item \textbf{The sensitivity of futures price to market changes can reflect the supply and demand relationship of carbon allowances more effectively than spot price, and ic an also effectively predict the trends in carbon price}.
	\item Some control variables are included in \cite{dong2022exploring}, say: 1. The level economic development (rapid economic debelopment will drive an increase in industrial production, and the prosperity resulting from the enhanced industrial production will lead to large carbon dioxided emissions, increasing the demand for carbon allowances); 2. Energy price (caron dioxide is mainly emitted from the use of primary energy. Changes in the price of fossil fuel energy affect the trend of carbon price; hence,  brent crude oil futures price is selected as a control variable); 3. Financial environment. The financial nature of carbon emission rights means that changes in the financial environment will affect carbon price trends (EU's interpeer dismantling interest rate as a control variable); 4. Clean development mechanism (CDM), which provides flexibility for countries to implement carbon emission reduction.
	
	\item Each year in which banked EUAs eceed 833 MtCO2, the number of auctioned EUAs next year is reduced. EUAs that are not auctioned are instead moved into the MSR. The revision of the MSR done in 2018 allowed to cancel allowances in the MSRT. This happens when the number of EUAs in the MSR exceeeds the number of auctioned EUAs of the previous year, in which case all EUAs in excess of this treshold are permanently canceld. Canceling EUAs in the MSR, which then cannot return to the market, effectively reduces the cumulative cap on emissions in the EU ETS; cumulative suply of EUAs has become \textbf{endogenous}. \cite{gerlagh2020covid}
	
	\item Prices in EU ETS also depend on expectations about regulation \cite{gerlagh2020covid}. For example, imagine some policy whose goal is to accelerate green transition and it's complementary to the ETS. If it leads to stronger emissions reductions in the short run than in the long run, the MSR boosts the effectiveness of this policy by taking out additional allowances. If an accelerated green transition reduces futures emissions more than current emissions, the MSR will partly reduce such policies, because decreased banking leads to reduced cancellation. Hence, cumulative amount of available allowances goes up.
	
	\item The  (upper) bound on the number of allowances held by the MSR originates from the February 2017 proposal of the Council of the European Union \cite{perino2017eu}. The MSR is an add-on to the EU ETS legislated in 2015 that starts operating in 2019 \cite{perino2018new}. The idea is to postpone the issue date of allowances as a function of the number of unused allowances in the market (bank). It is an autonomous version of the backloading implemented in Phase 3. It is designed to \emph{increase the short-term scarcity}. However, there's a \emph{water-bed effect}, meaning that the total number of allowances is not changed, i.e., extra-allowances are stored in the MSR and return at a later point in time, so it reduces scarcity in the medium term. This description applies to the very first version (i.e., 2015). However, in the 2017 revision some adjustments were agreed to make fundamental changes: the main of them is to have an impact on the long-term cap. Hence, the long-term cap is now a function of past and future market outcomes.
	
	\item The key feature of the 2017 agreement that changes the character of the EU ETS is the upper bound on allowances stored in the MSR.
	
	\item For every additional allowance banked, the number of allowanced placed in the MSR increases by 0.24 allowances (0.12 after 2023) in the first year, by $(1-0.24)\times 0.24 = 0.1824$ in the second year, and so on.
	
	\item The optimal policy mix crucially  depends on whether the cap in the EU ETS is fixed or a function of market outcomes \cite{perino2018new}.
	
	\item \textbf{Important}: ``A cap that is a function of market outcomes is sensible, but the rules should be simple and stable and their impacts predictable such that both market participants and regulators can understand them readily and respond accordingly. Such mechanisms do exist — but the new rules for Phase 4 are not among them.'' \cite{perino2018new}
	
	\item The EU ETS is designed to annually reducing carbon emission cap via the EU emission allowance (EUA) system. The idea is to provide a strong price signal for cost-effective greenhouse gass abatement in the European electric power sector, energy-intensive industry and the aviation sector \cite{bruninx2020long}.
	
	\item In 2018, the European Council decided to strengthen the ETS and MSR in three ways. 
	\begin{enumerate}[wide, itemsep=0cm, topsep=0cm, labelindent=2cm]
		\item From 2021 onward, the linnear reduction factor (LRF) of the emissions cap increases from 1.74\% to 2.2\%.
		\item From 2019 to 2023, the intake rate of the MSR doubles from 12\% to 24\%.
		\item From 2023 onward, the MSR cannot contain more allowances than the total number of allowances auctioned during the previous year.
		\item In adittion, the EU adopted a binding renewable energy target 32\% of the final energy use by 2030.
	\end{enumerate}
	\item Three different strnds are treated in \cite{bruninx2020long}:
	\begin{enumerate}[wide,itemsep=0cm, topsep=0cm, labelindent*=2cm]
		\item Dealing with the effect of EUA prices on CO$_2$ emissions in the electricity sector. Many papers have modelled the switching decisions in response to carbon prices and their interactions with renewable prices: from coal and oil to natural gas in the elctric power sectors lowers \cotwo emissions by 2\% to 19\% depending on the EUA price and the studied price \cite{bruninx2020long}. However, note that nowadays we have the case of Germany, which has closed all their nuclear power centrals and they have to burn coal instead of gas because of the extremely high prices of gas. 
		\item Effect of EUA prices on long-term investments in carbon abatement measures under the EU ETS. \cite{bruninx2020long} is the first paper to study the effect of an ETS on long-term electricity generation investment using an equilibrium model, and hence the first to assess the long-term quantitative effect of the strengthened MSR and increased LRF.
		\item Effect of an MSR in the EU ETs.
 	\end{enumerate}
 	\item Accurately capturing the costs of meeting the emissions cap today and in the future is critical in quantitative assessments of the impact of the MSR \cite{bruninx2020long}.
 	\item EUA price fluctuates in response to changing commodity prices, technological developments and policy decisions. 
 	\item The availability and costs of certain technologies, demand growth and discount rates affects the relative cost of meeting the emissions cap in the future, and hence, has an influence on  the profitability of banking allowances. This in turn affects the surplus today, the amount of allowances absorbed and cancelled by the MSR, and finally the cumulative emissions \cite{bruninx2020long}.
 	\item TNAc (Total Number of Allowances in Circulation) is a metric for the cumulative surplus between supply and demand for allowances \cite{perino2017eu, bruninx2020long}.
 	\item By setting a cap on emision permites and solving for the corresponding equilibrium, the computable general equilibrium (CGE) model finds the permit price and thus the marginal cost of abatement that corresponds to the baatement achived by the cap. \cite{landis2015final}
 	
 	\item Significant bidirectional dynamic causality between the EUA prices and the S\&P global clean energt intex \cite{lu2022transmission}.
 	\item The clean energy industries drive the development of the entire supply chain, which in turn affects the price of carbon emission rights.
\end{enumerate} 