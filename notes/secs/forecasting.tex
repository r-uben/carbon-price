The main idea is to use a reduced-form vector autoregression (VAR) that contains the fundamental drivers of the real price of oil,
\begin{equation}
	y_t = c + \sum_{i=1}^p \Phi_i y_{t-i} + \varepsilon_t,
\end{equation}
where $y_{t}\in\R^4$ is a vector of monthly data, $c\in\R^4$ is a vector of intercepts, and $\Phi_i\R^{4\times 4}$, for $i=1,\ldots, p$, are the coefficient matrices with $p$ indicating the number of lags. As always, $\varepsilon_t$ are the white-noise innovations.

The four variables are:
\begin{enumerate}[wide, itemsep=0cm, topsep=0cm]
	\item Percent change in global crude oil production;
	\item Estimate of the change in global crude oil inventories;
	\item Log of the real price of crude oil as measured by the US refiner acquisition cost of imported crude oil (RAC) deflated by the US consumer price index;
	\item Index of global real economic activity (REA).
\end{enumerate}

\subsection{Alternative indicators of Global Real Economic Activity}

\subsubsection{Real shipping cost factor}

\subsubsection{World industrial production}

\subsubsection{Weal commodity price factor}

\subsubsection{Global steel production factor}

\cite{ravazzolo2020world} suggests that steel is an important input for many industries including construction, transportation, and manufacturing, and that it is a relatively homogeneous commodity that is traded freely worldwide.

\begin{quote}
	World Steel Association: $\to$ aggregate measure of the level of steel production reported by member countries (consistent series only available since 1994).
\end{quote}

An increase in the number of steel-producing countiries leads to discrete jumps.

\cite{baumeister2020energy} propose an alternative way to construct a measure of global steel production that circumvents discontinuities and deals with the potential concern that world steel production is prone to idiosyncratic supply shocks in steel-producting countries.

\subsubsection{Summary}

All of the three alternative measures for global real economic activity do significantly better than the Killian index for forecasting either real RAC or Brent. However, none of these models beats the random walk at the intermediate horizons of 9 and 12 months ahead. Ovearall, forecasting oil prices with VAR models has become more difficult since 2010 and forecasting the real price of Brent poses additional challenges.


\subsection{Bayesian Shrinkage}

Whether Bayesian methods help reduce the MSPE of our VAR dorecasts by using informative priors that shrink our highly-parameterised unconstrained models toward a parsimonious benchmark, and thus reduce estimation uncertainty.

\textbf{Important:} They want to develop a forecasting model that emphasises the final demand for petroleum products. \textbf{Not only supply}. They replace global oil production with a measure of petroleum consumption. The broadest available measure at the global level is monthly total world consumption of liquid fuels provided in the Short-Term Energy outlook database of the US Energy Information Administration (only available from 1990.1 onward; they extend it back to 1982.1 using the growth rate of OECD petroleum consumption).

\subsection{Towards A New Indicator of Global Economic Conditions}

16 indicators (Table 7 in \cite{baumeister2020energy}). Variable selection is guided by four principles:

\begin{enumerate}[wide, itemsep=0cm, topsep=0cm]
	\item The different categories of data in order to span multiple dimensions of the global economy. Eight categories: real economic activity, commodity prices, financial indicators, transportation, uncertainty, expectations, weather, and energy-related measures.
	\item Each individual variable should matter for energy demand on economic grounds.
	\item It should have the broades possible coverage gepgraphically, conceptually and in time.
	
	\noindent \textbf{Comment:} It may not be necessary in our setting, because we have to focus in Europe and it may be more interesting to capture the particularities of Europe.
	\item The number of variables should be kept at a manageable size to ensure that the dataset can be easily updated in a real-time setting.
	
	For example: EIA's December 2019 \emph{Short-Term Energy outlook} considers developments of US real GDP, China's Purchasing Managers' Index, the S\& P 500 equity index, and the \textbf{copper-to-gold ratio as a measure of market sentiment} on global economic growth as important factors in their assessment of energy markets.
\end{enumerate}
